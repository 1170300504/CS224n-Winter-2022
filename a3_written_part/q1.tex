\titledquestion{Machine Learning \& Neural Networks}[8]
\begin{parts}

    \part[4] Adam Optimizer\newline
    Recall the standard Stochastic Gradient Descent update rule:
    \alns{
        \btheta &\gets \btheta - \alpha \nabla_{\btheta} J_{\text{minibatch}}(\btheta)
    }
    where $\btheta$ is a vector containing all of the model parameters, $J$ is the loss function, $\nabla_{\btheta} J_{\text{minibatch}}(\btheta)$ is the gradient of the loss function with respect to the parameters on a minibatch of data, and $\alpha$ is the learning rate.
    Adam Optimization\footnote{Kingma and Ba, 2015, \url{https://arxiv.org/pdf/1412.6980.pdf}} uses a more sophisticated update rule with two additional steps.\footnote{The actual Adam update uses a few additional tricks that are less important, but we won't worry about them here. If you want to learn more about it, you can take a look at: \url{http://cs231n.github.io/neural-networks-3/\#sgd}}

    \begin{subparts}
        \subpart[2] First, Adam adopts a commonly used technique called {\it momentum}, which keeps track of $\bm$, a rolling average of the gradients:
        \alns{
            \bm &\gets \beta_1\bm + (1 - \beta_1)\nabla_{\btheta} J_{\text{minibatch}}(\btheta) \\
            \btheta &\gets \btheta - \alpha \bm
        }
        where $\beta_1$ is a hyperparameter between 0 and 1 (often set to 0.9). Briefly explain in 2-4 sentences (you don't need to prove mathematically, just give an intuition) how using $\bm$ stops the updates from varying as much and why this low variance may be helpful to learning, overall.

        \ifans{
            It is obvious that only a small portion of $\bm$ may get affected in a single update, and therefore
            the {\it momentum} somehow reduces the amplitude of oscillations and avoids excessive change of $\btheta$ in a single step.
            This low variance helps maintain the efficiency of gradient descent, leading to faster convergence.
        }\newline

        \subpart[2] Adam extends the idea of {\it momentum} with the technique of {\it adaptive learning rates} by keeping track of  $\bv$, a rolling average of the magnitudes of the gradients:
        \alns{
            \bm &\gets \beta_1\bm + (1 - \beta_1)\nabla_{\btheta} J_{\text{minibatch}}(\btheta) \\
            \bv &\gets \beta_2\bv + (1 - \beta_2) (\nabla_{\btheta} J_{\text{minibatch}}(\btheta) \odot \nabla_{\btheta} J_{\text{minibatch}}(\btheta)) \\
            \btheta &\gets \btheta - \alpha \bm / \sqrt{\bv}
        }
        where $\odot$ and $/$ denote elementwise multiplication and division (so $\bz \odot \bz$ is elementwise squaring) and $\beta_2$ is a hyperparameter between 0 and 1 (often set to 0.99). Since Adam divides the update by $\sqrt{\bv}$, which of the model parameters will get larger updates? Why might this help with learning?

        \ifans{
            Model parameters with a scarce history of updates will get larger updates. This normalizes the update steps, avoids overshooting or a monotonically decreasing learning rate.
        }\newline

    \end{subparts}


    \part[4]
    Dropout\footnote{Srivastava et al., 2014, \url{https://www.cs.toronto.edu/~hinton/absps/JMLRdropout.pdf}} is a regularization technique. During training, dropout randomly sets units in the hidden layer $\bh$ to zero with probability $p_{\text{drop}}$ (dropping different units each minibatch), and then multiplies $\bh$ by a constant $\gamma$. We can write this as:
    \alns{
        \bh_{\text{drop}} = \gamma \bd \odot \bh
    }
    where $\bd \in \{0, 1\}^{D_h}$ ($D_h$ is the size of $\bh$)
    is a mask vector where each entry is 0 with probability $p_{\text{drop}}$ and 1 with probability $(1 - p_{\text{drop}})$. $\gamma$ is chosen such that the expected value of $\bh_{\text{drop}}$ is $\bh$:
    \alns{
        \mathbb{E}_{p_{\text{drop}}}[\bh_\text{drop}]_i = h_i \text{\phantom{aaaa}}
    }
    for all $i \in \{1,\dots,D_h\}$.
    \begin{subparts}
        \subpart[2]
        What must $\gamma$ equal in terms of $p_{\text{drop}}$? Briefly justify your answer or show your math derivation using the equations given above.\\
        \ifans{
            The probability of retaining a unit is $ (1-p_{\text{drop})} $. Therefore, the expected value of $ \bd \odot \bh $ is $(1-p_{\text{drop}})$.
            So $\gamma$ must equal to $ \frac{1}{1-p_{\text{drop}}} $.
        }\newline

        \subpart[2] Why should dropout be applied during training? Why should dropout \textbf{NOT} be applied during evaluation? (Hint: it may help to look at the paper linked above in the write-up.)
        \ifans{
            Applying dropout to a neural network during training somehow samples a ``thinned" network from it, and therefore
            prevents overfitting. However, there is no need to prevent overfitting during evaluation. Also, apply dropout during evaluation
            undermines the consistency of the model output.
        }\newline

    \end{subparts}
\end{parts}
